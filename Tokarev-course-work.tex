\documentclass[bachelor, och, coursework]{SCWorks}
% параметр - тип обучения - одно из значений:
%    spec     - специальность
%    bachelor - бакалавриат (по умолчанию)
%    master   - магистратура
% параметр - форма обучения - одно из значений:
%    och   - очное (по умолчанию)
%    zaoch - заочное
% параметр - тип работы - одно из значений:
%    referat    - реферат
%    coursework - курсовая работа (по умолчанию)
%    diploma    - дипломная работа
%    pract      - отчет по практике
% параметр - включение шрифта
%    times    - включение шрифта Times New Roman (если установлен)
%               по умолчанию выключен
\usepackage{subfigure}
\usepackage{tikz,pgfplots}
\pgfplotsset{compat=1.5}
\usepackage{float}

%\usepackage{titlesec}
\setcounter{secnumdepth}{4}
%\titleformat{\paragraph}
%{\normalfont\normalsize}{\theparagraph}{1em}{}
%\titlespacing*{\paragraph}
%{35.5pt}{3.25ex plus 1ex minus .2ex}{1.5ex plus .2ex}

\titleformat{\paragraph}[block]
{\hspace{1.25cm}\normalfont}
{\theparagraph}{1ex}{}
\titlespacing{\paragraph}
{0cm}{2ex plus 1ex minus .2ex}{.4ex plus.2ex}

% --------------------------------------------------------------------------%


\usepackage[T2A]{fontenc}
\usepackage[utf8]{inputenc}
\usepackage{graphicx}
\graphicspath{ {./images/} }
\usepackage{tempora}

\usepackage[sort,compress]{cite}
\usepackage{amsmath}
\usepackage{amssymb}
\usepackage{amsthm}
\usepackage{fancyvrb}
\usepackage{listings}
\usepackage{listingsutf8}
\usepackage{longtable}
\usepackage{array}
\usepackage[english,russian]{babel}

% \usepackage[colorlinks=true]{hyperref}
\usepackage{url}

\usepackage{underscore}
\usepackage{setspace}
\usepackage{indentfirst} 
\usepackage{mathtools}
\usepackage{amsfonts}
\usepackage{enumitem}
\usepackage{tikz}
% \usepackage{libertine}
% \usepackage{listings}
% \usepackage{listingsutf8}
% \usepackage{xcolor}
% Настройка шрифта Courier New
% \usepackage{courier} % Подключение шрифта Courier New
\renewcommand{\ttdefault}{pcr} % Использование Courier New для моноширинного текста

\definecolor{codegreen}{rgb}{0,0.6,0}
\definecolor{codegray}{rgb}{0.5,0.5,0.5}
\definecolor{codepurple}{rgb}{0.58,0,0.82}
\definecolor{backcolour}{rgb}{1,1,1}

\lstdefinestyle{mystyle}{
    language = Python, % Указываем язык программирования (например, Python)
    extendedchars = true,   % Поддержка расширенных символов
    inputencoding=utf8,     % Кодировка входного файла
    literate={á}{{\'a}}1 {é}{{\'e}}1 {í}{{\'i}}1 {ó}{{\'o}}1 {ú}{{\'u}}1, % Дополнительная поддержка диакритики
    backgroundcolor=\color{backcolour},   
    commentstyle=\color{codegreen}\itshape, % Комментарии будут зелеными и курсивными
    keywordstyle=\color{magenta},
    numberstyle=\tiny\color{codegray},
    stringstyle=\color{codepurple},
    basicstyle=\ttfamily\footnotesize, % Используем шрифт Courier New, размер 11 пт
    frame=single,                      % Рамка вокруг листинга
    rulesepcolor=\color{black!30},     % Цвет рамки
    framerule=0.4pt,                   % Толщина рамки
    aboveskip=10pt,                    % Отступ сверху от листинга
    belowskip=10pt,                    % Отступ снизу от листинга
    breakatwhitespace=false,           % Перенос строк без пробелов
    breaklines=true,                   % Разрешаем перенос строк
    captionpos=b,                      % Заголовок под листингом
    keepspaces=true,                   % Сохраняем пробелы
    numbers=left,                      % Нумерация слева
    numbersep=5pt,                     % Расстояние между номерами и кодом
    showspaces=false,                  % Не показывать пробелы
    showstringspaces=false,            % Не показывать пробелы в строках
    showtabs=false,                    % Не показывать табуляции
    tabsize=2                          % Размер табуляции
}

\lstset{style=mystyle}

% \usepackage{minted}

\newcommand{\eqdef}{\stackrel {\rm def}{=}}
\newcommand{\specialcell}[2][c]{%
\begin{tabular}[#1]{@{}c@{}}#2\end{tabular}}

\renewcommand\theFancyVerbLine{\small\arabic{FancyVerbLine}}

\newtheorem{lem}{Лемма}

\begin{document}

% Кафедра (в родительном падеже)
\chair{теоретических основ компьютерной безопасности и криптографии}

% Тема работы
\title{Разработка программы для выявления RDP-сессий с использованием нейронных сетей}

% Курс
\course{6}

% Группа
\group{631}

% Факультет (в родительном падеже) (по умолчанию "факультета КНиИТ")
\department{факультета КНиИТ}

% Специальность/направление код - наименование
%\napravlenie{09.03.04 "--- Программная инженерия}
%\napravlenie{010500 "--- Математическое обеспечение и администрирование информационных систем}
%\napravlenie{230100 "--- Информатика и вычислительная техника}
%\napravlenie{231000 "--- Программная инженерия}
\napravlenie{10.05.01 "--- Компьютерная безопасность}

% Для студентки. Для работы студента следующая команда не нужна.
% \studenttitle{Студентки}

% Фамилия, имя, отчество в родительном падеже
\author{Токарева Никиты Сергеевича}

% Заведующий кафедрой
\chtitle{} % степень, звание
\chname{Абросимов М. Б.}

%Научный руководитель (для реферата преподаватель проверяющий работу)
\satitle{доцент} %должность, степень, звание
\saname{Гортинский А. В.}

% Руководитель практики от организации (только для практики,
% для остальных типов работ не используется)
% \patitle{к.ф.-м.н.}
% \paname{С.~В.~Миронов}

% Семестр (только для практики, для остальных
% типов работ не используется)
%\term{8}

% Наименование практики (только для практики, для остальных
% типов работ не используется)
%\practtype{преддипломная}

% Продолжительность практики (количество недель) (только для практики,
% для остальных типов работ не используется)
%\duration{4}

% Даты начала и окончания практики (только для практики, для остальных
% типов работ не используется)
%\practStart{30.04.2019}
%\practFinish{27.05.2019}

% Год выполнения отчета
\date{2023}

\maketitle

% Включение нумерации рисунков, формул и таблиц по разделам
% (по умолчанию - нумерация сквозная)
% (допускается оба вида нумерации)
% \secNumbering

%-------------------------------------------------------------------------------------------

\tableofcontents

\intro
В условиях современной цифровой экономики удалённый доступ к информационным системам играет важную роль в обеспечении удобства, 
гибкости и эффективности работы. Одним из наиболее популярных протоколов для удалённого управления компьютерами является Remote 
Desktop Protocol (RDP). Этот протокол широко используется в корпоративных и частных сетях для предоставления безопасного доступа 
к рабочим станциям, серверам и другим устройствам.


Однако вместе с удобством использования RDP остаётся объектом пристального внимания со стороны злоумышленников. Уязвимости протокола 
часто используются для реализации атак, включая подбор паролей методом перебора (brute force), использование слабых паролей, кражу 
данных и распространение вредоносного ПО. В таких условиях задача мониторинга и выявления RDP-трафика в общем потоке сетевых данных 
приобретает особую актуальность.


Основной целью данной работы является разработка программы для автоматического выявления RDP-сессий в сетевом трафике с использованием 
нейронных сетей, а именно модели LSTM (Long Short-Term Memory). Эта модель, благодаря своей способности обрабатывать последовательности данных, 
предоставляет мощный инструмент для анализа характеристик сетевого трафика и классификации его по типу.

Для достижения цели были поставлены следующие задачи:

\begin{itemize}
  \item изучение особенностей протокола RDP и его признаков в сетевом трафике;
  \item разработка метрик для анализа и классификации RDP-трафика;
  \item использование модели LSTM для решения задачи классификации;
  \item реализация программного продукта, способного анализировать сетевой трафик в режиме реального времени и выявлять RDP-сессии;
  \item тестирование программы и оценка её эффективности.
\end{itemize}

Результаты данной работы могут быть полезны для специалистов в области информационной безопасности, системных администраторов, а также 
разработчиков инструментов мониторинга и анализа сетевого трафика.

% \begin{figure}[H]
%   \centering
%   \includegraphics[width=0.9\textwidth]{pics/http2.jpg}
%   \caption{}
%   \label{}
% \end{figure}
\section{Протокол RDP: место в сетях и особенности реализации}

Концепция удалённого доступа к вычислительным ресурсам зародилась в стремлении управлять компьютерами и серверами, находящимися на расстоянии. 
Это позволило пользователям быть независимыми от физического местоположения оборудования, обеспечивая гибкость и эффективность в использовании 
вычислительных мощностей.

Протокол удаленного рабочего стола (Remote Desktop Protocol, RDP) был разработан корпорацией Microsoft в конце 1990-х годов как решение для 
удаленного доступа к компьютерам и серверам. Истоки технологии удаленного доступа уходят в 90-е годы XX века, когда компании начали создавать 
протоколы для управления вычислительными системами на расстоянии. Одним из первых таких протоколов был Citrix ICA, который стал основой для 
создания RDP. Microsoft внедрила эту технологию в свои продукты, начиная с Windows NT Terminal Server 4.0, что значительно упростило удалённую работу.

\subsection{Обзор протокола RDP}

Идея создания RDP была основана на стремлении обеспечить пользователям возможность удалённого управления вычислительными ресурсами, независимо от 
их физического расположения. Эта технология стала важной частью корпоративных сетей, позволив администраторам эффективно управлять серверами, а 
пользователям -- работать с удалёнными рабочими столами. Вначале RDP был ориентирован на работу в режиме точка-точка, но со временем получил 
поддержку многоточечных соединений, что сделало его более универсальным инструментом для совместной работы и администрирования \cite{rdp2}.

RDP является расширением семейства протоколов T.120 и базируется на стандарте T.Share. Его ключевые особенности:

\begin{enumerate}
  \item Многоканальная архитектура: RDP поддерживает до 64 000 виртуальных каналов для передачи различных типов данных, таких как:
  
  \begin{itemize}
    \item данные презентации;
    \item управление периферийными устройствами;
    \item лицензирование;
    \item зашифрованные команды клавиатуры и мыши.
    
  \end{itemize}
  
  \item Безопасность: протокол включает механизмы шифрования, обеспечивающие защиту передаваемых данных. Это делает RDP надёжным выбором для корпоративных сетей;
  
  \item Универсальность: RDP изначально был разработан для работы с различными сетевыми топологиями, включая ISDN, POTS и TCP/IP. Современные версии фокусируются на TCP/IP, обеспечивая широкую совместимость;
  
  \item Оптимизация сетевого трафика: протокол использует компрессию данных и механизмы кадрирования для минимизации сетевых задержек;
  
  \item Поддержка мультимедиа: современные версии RDP включают функции передачи аудио, видео и взаимодействия с периферийными устройствами, такими как принтеры и USB-устройства.
\end{enumerate}


\subsection{Место RDP в стеке протоколов}

Стек протоколов представляет собой набор взаимосвязанных стандартов и правил, определяющих взаимодействие между различными уровнями сетевой 
архитектуры. Каждый уровень отвечает за выполнение определённых функций, начиная от передачи данных через физические среды до обеспечения 
высокого уровня абстракции для приложений. Существует несколько моделей сетевых взаимодействий, каждая из которых предлагает свой подход 
к организации передачи данных между устройствами в сети.

Наиболее известным примером стека протоколов является модель OSI (Open Systems Interconnection), которая состоит из семи уровней:

\begin{enumerate}
  \item Физический уровень, который отвечает за передачу последовательности битов через канал связи.
  \item Канальный уровень, где осуществляется разбиение данных на <<кадры>>, размер которых обычно достигает
  от несколько сотен до нескольких тысяч байтов.
  \item Сетевой уровень, на котором осуществляется структуризация и маршрутизация пакетов от отправителя к получателю.
  \item Транспортный уровень, функцией которого является передача надежных последовательностей данных произвольной
  длины через коммуникационную сеть от отправителя к получателю.
  \item Сеансовый уровень, на котором происходит поддержка сессии связи, уп-\\*равление взаимодействием между приложениями.  
  \item Уровень представления, который представляет данные в понятном для какой-либо конкретной машины виде.
  \item Прикладной уровень, предоставляющий набор интерфейсов для взаимодействия пользовательских процессов с сетью \cite{osi-model}.
\end{enumerate}

Вследствие этого, RDP является непосредственно протоколом прикладного уровня модели OSI, наряду с HTTP, FTP, SSH и многими другими. Стоит отметить, что 
в основном OSI применяется преимущественно в образовательных и теоретических целях. На практике используется модель TCP/IP, которая отражает
реальную организацию современных компьютерных сетей \cite{stack}. Она была разработана как часть стека протоколов, лежащих в основе интернета. 
Название TCP/IP связано с двумя ключевыми протоколами этого семейства — Transmission Control Protocol (TCP) и Internet Protocol (IP). Именно они 
были впервые разработаны и задокументированы в этом стандарте. Иногда эту модель называют моделью DOD (Department of Defense).
TCP/IP используется повсеместно, поскольку большинство современных протоколов (HTTP, FTP, RDP) работают в рамках этой модели. Модель TCP/IP 
разделяет сетевое взаимодействие на четыре уровня, каждый из которых выполняет определённые функции:

\begin{enumerate}
    \item Сетевой интерфейс/канальный уровень (объединяет уровни 1 и 2 OSI): отвечает за физическую передачу данных.
    \item Сетевой уровень (аналог уровня 3 OSI): используется для маршрутизации данных (например, протокол IP).
    \item Транспортный уровень (аналог уровня 4 OSI): обеспечивает надежность передачи данных с помощью протоколов TCP и UDP.
    \item Прикладной уровень (соответствует уровням 5, 6 и 7 OSI): поддерживает приложения и службы, такие как HTTP, FTP, RDP и т.д.
\end{enumerate}

Основное преимущество TCP/IP в том, что это практическая модель, на основе которой построен интернет. Все современные сети работают 
с протоколами TCP/IP, поэтому в реальных системах она имеет приоритет.

Здесь важно понимать, что в компьтерных сетях все данные инкапсулируются и декапсулируются на различных уровнях модели TCP/IP.
Процесс инкапсуляции происходит так, что на каждом уровне стека 
протоколы добавляют свои заголовки: от уровня приложений до канального уровня. На этапе приема данные декапсулируются в обратном порядке, начиная 
с канального уровня и заканчивая уровнем приложений. 

Исходя из описания уровней, RDP в модели TCP/IP также относится к прикладному уровню. Поэтому, данные RDP сначала инкапсулируются в сегменты TCP, 
которые затем оборачиваются в IP-пакеты и кадры Ethernet для передачи по сети. На принимающем устройстве данные декапсулируются в обратном 
порядке, начиная с канального уровня и заканчивая уровнем приложений, уровнем RDP.

Для понимания процесса передачи данных от отправителя к получателю важно рассмотреть структуру пакетов, используемых в сетевых взаимодействиях. Далее
будет рассмотрена четырехуровневая структура модели TCP/IP от канального до прикладного. 


\subsubsection{Структура Ethernet-кадра}

На канальном уровне передача данных осуществляется с использованием Ethernet-кадра. Его структура представлена на рисунке \ref{eth-frame}.

\begin{figure}[H]
  \centering
  \includegraphics[width=0.9\textwidth]{pics/eth-frame.jpg}
  \caption{Структура Ethernet кадра}
  \label{eth-frame}
\end{figure}

Ключевыми полями Ethernet-кадра являются:

\begin{itemize}
  \item MAC-адреса источника и назначения -- позволяют определить отправителя и получателя на канальном уровне.
  \item Поле <<Тип>> -- указывает номер инкапсулированного сетевого протокола (например, IPv4 или IPv6).
  \item Поле данных -- содержит инкапсулированные данные более высокого уровня, включая сетевые пакеты.
\end{itemize}

\subsubsection{Структура IPv4-заголовка}

Для передачи TCP-сегментов на сетевом уровне используется IP-протокол. В данной работе рассматривается версия IPv4, 
так как её возможностей достаточно для анализа RDP-трафика. Структура IPv4-заголовка представлена на рисунке \ref{ipv4-header}.

\begin{figure}[H]
  \centering
  \includegraphics[width=0.9\textwidth]{pics/ipv4-header.png}
  \caption{Структура IPv4-заголовка}
  \label{ipv4-header}
\end{figure}

Особо важны следующие поля IPv4-заголовка:

\begin{itemize}
  \item IP-адреса отправителя и получателя -- позволяют идентифицировать ус-\\*тройства, участвующие в обмене данными.
  \item Поле <<Протокол>> -- определяет протокол транспортного уровня. Например, значение 6 указывает на TCP, а 17 -- на UDP.
\end{itemize}

\subsubsection{Структура заголовков TCP и UDP}

Известно, что протокол RDP взаимодействует с транспортным уровнем через TCP и UDP.
В основном данный протокол использует TCP для установления надёжного соединения и передачи данных, что позволяет обеспечить стабильное взаимодействие 
между клиентом и сервером удалённого рабочего стола. Сама структура TCP-заголовка показана на рисунке \ref{tcp-header}.

  \begin{figure}[H]
    \centering
    \includegraphics[width=0.9\textwidth]{pics/tcp-segment.png}
    \caption{Структура TCP-заголовка}
    \label{tcp-header}
  \end{figure}

В данном сегменте одними из интересных полей является информация о портах отправителя и получателя. Стоит отметить, что при подключении к 
удаленному рабочему столу по умолчанию используется порт 3389, что помогает идентифицировать RDP-трафик.

Не менее интересным полей является поле <<Размер окна>> -- это объем данных приема (в байтах), которые можно буферизировать во время подключения. 
Узел отправки может отправлять только этот объем данных, прежде чем он должен ожидать подтверждения и обновления окна от принимающего узла \cite{winsize}.
Другими словами, данная величина указывает количество байт, которое может быть отправлено без подтверждения от получателя до того момента, когда 
необходимо будет ожидать нового подтверждения о получении данных.

Также важное значение имеют флаги, содержащиеся в поле флагов. В нем хранятся следующие управляющие биты:

\begin{enumerate}
  \item NS -- одноразовая сумма (Nonce Sum). По-прежнему является экспериментальным флагом, используемым для защиты от случайного
  злонамеренного сокрытия пакетов от отправителя \cite{tcpflags}. Используется для улучшения работы механизма явного уведомления 
  о перегрузке (Explicit Congestion Notification, ECN).
  \item CWR -- окно перегрузки уменьшено (Congestion Window Reduced). 
  Данный флаг устанавливается (принимает значение равной единице) отправителем, чтобы показать, что TCP-фрагмент был
  получен с установленным полем ECE.
  \item ECE -- ECN-Эхо (ECN-Echo). Этот флаг показывает, поддерживает ли TCP-отправитель ECN.
  \item URG -- устанавливается, если необходимо передать ссылку на поле указателя срочности (Urgent pointer).
  \item ACK -- флаг подтверждения используется для подтверждения успешного получения пакета.
  \item PSH -- инструктирует получателя протолкнуть данные, накопившиеся в приемном буфере, в приложение пользователя.
  \item RST -- флаг сброса отправляется от получателя к отправителю, когда пакет отправляется на конкретный хост, который этого не ожидал.
  \item SYN -- начинает соединение и синхронизирует порядковые номера. Первый пакет, отправленный с каждой стороны, должен в обязательном порядке иметь установленным этот флаг.
  \item FIN -- означает, что данных от отправителя больше нет. Поэтому он используется в последнем пакете, отправленном отправителем.
\end{enumerate}

Они используются для управления соединением, передачи данных и завершения сессий.
Благодаря этим флагам можно определить текущее состояние соединения и характер обмена данными.

Поле <<Данные>> содержит информацию, которую передает приложение, использующее транспортный уровень. То есть, данные, которые 
находятся в этом поле, принадлежат уровню выше транспортного -- прикладному уровню. Таким образом, именно в этом поле хранятся данные о протоколе RDP.

Начиная с версии RDP 8.0 (введенной с Windows Server 2012 и Windows 8), протокол стал поддерживать UDP (англ. User Datagram Protocol -- протокол пользовательских датаграмм)
как дополнительный транспортный протокол для оптимизации работы в условиях нестабильных сетей \cite{udpseg}. Поэтому на структуру UDP протокола тоже стоит обратить внимание.
В отличие от TCP протокол UDP имеет минималистичный заголовок. Его можно увидеть на следующем рисунке.


\begin{figure}[H]
  \centering
  \includegraphics[width=0.9\textwidth]{pics/udp-segment.png}
  \caption{Структура UDP-заголовка}
  \label{udp-header}
\end{figure}

Из полезной информации можно выделить только номера портов и поле <<Данные>>.

Стоит отметить, что RDP использует UDP в случаях, когда нужно минимизировать задержки и повысить производительность, особенно для мультимедийных 
задач или нестабильных сетевых условий \cite{udpseg}. Однако TCP остаётся основным протоколом для критически важных операций, таких как 
управление соединением и передача данных с высокой надежностью.

% Далее необходимо рассмотреть методы обнаружения RDP-трафика, которые позволят эффективно выявить наличие использования RDP-протокола.

Рассмотрение структуры пакетов, представленной в данном разделе, позволяет получить общее представление о принципах организации передачи данных в сетях и месте 
протокола RDP в стеке сетевых протоколов. Однако для решения задачи идентификации RDP-трафика среди общего потока данных возникают некоторые сложности.

Во-первых, RDP-трафик обычно передаётся в зашифрованном виде. Даже если программно получить доступе к полю данных в заголовках TCP и UDP, то расшифровка 
содержимого без наличия ключа шифрования становится практически невозможной. Это делает анализ RDP-заголовков недостаточно эффективным для точного 
определения наличия RDP-трафика.

Во-вторых, использование портов для идентификации RDP также не является надёжным методом. Злоумышленники могут изменить стандартный порт, 
применяемый для RDP-соединений, с целью избежать обнаружения. Более того, если на одном устройстве функционирует несколько RDP-сессий, для каждой 
из них могут использоваться разные порты, что дополнительно усложняет задачу анализа.

Таким образом, возникает вопрос: каким образом можно надёжно отличить трафик протокола удалённого рабочего стола от других протоколов, наблюдаемых в сетевой среде?

Ответ на данный вопрос будет рассмотрен в следующем разделе, где проанализированы ключевые метрики и особенности, характерные для RDP-трафика.

\section{Особенности и характерные признаки RDP-трафика}

В рамках моей предыдущей курсовой работы, посвящённой теме <<Статистический анализ сетевого трафика для обнаружения активной RDP-сессии>>, 
были изучены и применены различные методы статистического анализа для выявления RDP-трафика. В частности, рассматривались такие подходы, как:  

\begin{itemize}
  \item анализ распределения временных интервалов между пакетами;  
  \item исследование распределения размера пакетов;  
  \item анализ частоты флагов PSH;  
  \item нахождение отношения входящего и исходящего трафика.  
\end{itemize}

Результаты курсовой работы показали, что использование этих методов в совокупности позволяет с высокой степенью вероятности идентифицировать 
RDP-трафик в сетевой среде.  


В рамках настоящего исследования статистические методы рассматриваются более детально. Они дополняются новыми метриками, 
учитывающими динамические особенности RDP-сессий. Эти признаки используются в качестве входных данных для обучения нейронной сети, что позволяет 
значительно повысить точность и надёжность обнаружения RDP.

В данном разделе подробно описываются признаки, характерные для RDP-трафика и способы расчета метрик на основе этих признаков.
Все представленные метрики сопровождаются формулами и пояснениями их реализации в программной системе. Стоит отметить, что вся информация, 
используемая при вычислении метрик, собирается самой программой. Программа извлекает из перехваченных пакетов исключительно те данные, которые 
требуются для расчёта метрик.

\subsection{Признаки на основе анализа временных интервалов между пакетами}

Анализ временных интервалов между пакетами может помочь выявить RDP-сессии. Как правило, промежутки между пакетами RDP-трафика короче, 
чем у других видов трафика. Это объясняется тем, что протокол RDP рассчитан на передачу данных в реальном времени и требует быстрой 
доставки информации для поддержания стабильного соединения с удаленным рабочим столом. Соответственно, если в сети наблюдается 
большое количество пакетов с короткими временными интервалами, это может свидетельствовать о наличии активной RDP-сессии. Тем 
не менее, важно помнить, что существуют и другие виды трафика, использующие высокие скорости передачи данных и характеризующиеся 
малыми интервалами между пакетами. Поэтому вычисляется несколько метрик, получаемые из промежутков временных интервалов, чтобы улучшить
процесс определения RDP-трафика.

\subsubsection{Вычисление задержки и стандартного отклонения временных интервалов между пакетами}

Под задержкой будем понимать среднее время, которое проходит между отправкой и получением пакетов. Пусть $ t_1, t_2, \dots, t_n $, 
где $ t_i $ -- это временная метка $ i $-го пакета. Разработанная программа запоминает все эти временные метки и находит разницу ($\Delta t_j$) 
между двумя пакетами следующим образом:
\begin{equation}
  \Delta t_j = t_{i+1} - t_{i}, \text{ для всех } i = \overline{1, n - 1}
\end{equation}

Тогда задержка вычисляется по следующей формуле:

\begin{equation}
  \mu_L = \frac{1}{n - 1} \sum_{i=1}^{n-1} \Delta t_i
\end{equation}

Исходя и формулы (2) можно заметить, что задержка -- это среднее значение временных интервалов между пакетами.

Вычисление стандартного отклонения, получается из значения задержки по следующей формуле:

\begin{equation}
  \sigma = \sqrt{\frac{1}{n-1} \sum_{i=1}^{n-1} (\Delta t_i - \mu_L)^2}
\end{equation}

  Среднее значение (задержка) и стандартное отклонение позволяет оценить стабильность интервалов времени 
  и, следовательно, стабильность передачи данных, что является одной из характерных черт RDP-трафика.

\subsubsection{Вычисление среднего джиттера и медианы временных интервалов}

   Джиттером называется изменение временной задержки между передачей данных и их получением по сетевому соединению.
   Проще говоря, джиттер отражает нестабильность задержки. Данная величина вычисляется по следующей формуле:

   \begin{equation}
      J_i = |\Delta t_{i+1} - \Delta t_i|, \quad \text{для } i = \overline{1, n - 2}.
   \end{equation}

  Тогда средний джиттер вычисляется следующим образом:

   \begin{equation}
    \mu_J = \frac{1}{n-2} \sum_{i=1}^{n-2} J_i
   \end{equation}

   Этот показатель характеризует, насколько стабильны интервалы между пакетов, что важно для оценки качества RDP-сессий, 
   поскольку низкий джиттер необходим для стабильной работы удаленного рабочего стола. 
   
   Исходя из формул вычисления задержки и джиттера может показаться, что эти величины похожи, но это не так. Они измеряют разные 
   аспекты передачи данных, и каждая из этих метрик имеет свое значение в контексте анализа RDP-трафика. Задержка показывает, как долго 
   передаются пакеты, а джиттер измеряет колебания этого времени.

   Медиана временных интервалов является также полезной метрикой, которая может помочь сгладить влияние выбросов и экстремальных значений. 
   Медиана представляет собой среднее значение в отсортированном ряду интервалов времени. Вследствие этого временные метки $t_i$ ($i = \overline{1, n}$) 
   перехваченных пакетов сортируются по возрастанию.
   
   Далее если количество интервалов $ n $ нечетное, то медианой будет центральный элемент отсортированного ряда. 
   Если $ n $ четное, медианой будет среднее значение двух центральных элементов.

   Таким образом медиана вычисляется следующим образом:

   \begin{equation}
    m_t = 
      \begin{cases}
        t_{n / 2}, & \text{если n нечетное}\\
        (t_{n / 2} + t_{n / 2 + 1}) / 2, & \text{если n четное}
      \end{cases}
   \end{equation}
  

\subsection{Признаки, связанные с количественными характеристиками трафика}


Помимо анализа временных интервалов, важную информацию о характере трафика можно получить из количественных характеристик передаваемых пакетов.
Анализ объёма и распределения трафика позволяет выявить ключевые закономерности, присущие протоколу RDP. Такие признаки, как соотношение 
входящего и исходящего трафика, распределение объёмов между UDP- и TCP-пакетами, а также средний размер передаваемых пакетов, отражают 
особенности поведения RDP-сессий. Каждый сетевой протокол обладает уникальным паттерном обмена данными, и изучение этих характеристик 
играет важную роль в их идентификации.

\subsubsection{Вычисление отношения объема входящего на исходящий трафик}

Разработанная программа сохраняет пакеты относительно инициатора подключения (клиента) и целевого устройства (сервера). В процессе работы накапливаются 
две величины: объем исходящего трафика ($V_{src}$) — количество данных, отправляемых клиентом, и объем входящего трафика ($V_{dest}$) — количество данных, 
отправляемых сервером.

Отношение объема входящего на исходящий трафик вычисляется по формуле:

\begin{equation}
  r_{dest/src}^{init} = \frac{V_{dest}}{V_{src}}
\end{equation}

Аналогично отношение объема трафика относительно сервера вычисляется как:

\begin{equation}
  r_{dest/src}^{targ} = \frac{V_{src}}{V_{dest}}
\end{equation}

Эти две величины позволяют оценить симметричность обмена данными между клиентом и сервером. Для протокола RDP характерен асимметричный обмен 
трафиком: большая часть данных передается от сервера к клиенту, что обусловлено характером работы протокола, где сервер передает графическую 
информацию, а клиент отправляет команды управления.

Рассмотрение обеих величин важно, поскольку оно обеспечивает всесторонний анализ взаимодействия между клиентом и сервером. Например, высокое 
значение $r_{dest/src}^{init}$ (преобладание входящего трафика) может указывать на активность RDP-сессии, тогда как отклонения в этих показателях 
могут сигнализировать о другом типе соединения. Метрика особенно полезна в контексте других признаков, так как её устойчивость к шуму и вариативность 
в зависимости от типа трафика делает её важным элементом для классификации с использованием нейронной сети.

\subsubsection{Вычисление отношения объема UDP-трафика и TCP-трафика}

В предыдущем разделе упоминалось, что начиная с версии 8.0, протокол RDP поддерживает передачу данных по UDP. Это особенно заметно в 
сетевом обмене между компьютерами с установленной операционной системой Windows. По протоколу UDP в основном передаются графические данные, 
а также действия, совершаемые мышью и клавиатурой, что делает его важной частью работы RDP. Учитывая это, исключать анализ UDP-трафика из 
общей картины трафика было бы неправильно.

Разработанная программа отслеживает объем переданных пакетов по протоколам UDP и TCP, а также вычисляет их соотношение по следующей формуле:

\begin{equation}
  r_{udp/tcp} = \frac{V_{udp}}{V_{tcp}}
\end{equation}

Таким образом, данная метрика отражает баланс использования двух транспортных протоколов в рамках одного соединения. Для RDP характерно 
заметное преобладание трафика по протоколу TCP, однако наличие UDP-трафика с характерным объемом может служить дополнительным индикатором 
активности RDP-сессии.

Эта метрика становится особенно эффективной в сочетании с другими признаками, улучшая точность классификации сетевого трафика и выявления RDP.

\subsubsection{Вычисление среднего значения объема пакетов}

При работе протоколов прикладного уровня данные передаются пакетами различных объемов. 
Размер пакетов может варьироваться в зависимости от характера передаваемой информации: от небольших управляющих сообщений до крупных сегментов данных
Исследование размеров передаваемых пакетов может оказаться полезным инструментом при анализе сетевого 
трафика, особенно когда речь идет об определении аномалий или подозрительных активностей.


В разработанной программе анализируется размер блока «Данные», содержащегося в заголовках протоколов TCP и UDP. В процессе перехвата 
трафика программа сохраняет размеры полезной нагрузки каждого пакета: \\ $p_{s_1}, p_{s_2}, \dots, p_{s_n}$. На основании этих данных 
вычисляется среднее значение объема передаваемых пакетов:

\begin{equation}
  \mu_s = \frac{1}{n} \sum_{i=1}^{n} \Delta p_{s_i}
\end{equation}

Эта метрика предоставляет информацию о характере передаваемого трафика. Для протокола RDP характерен обмен пакетами малого и среднего 
размера, отражающий отправку управляющих сигналов, таких как движения мыши, нажатия клавиш, или обновления небольших частей экрана. В 
отличие от этого, потоки мультимедиа или файлы, передаваемые другими протоколами, обычно характеризуются значительно большими средними 
размерами пакетов.

Значимость среднего значения объема пакетов заключается в его способности идентифицировать типичные характеристики RDP-сессий и исключать 
несоответствующий трафик, тем самым улучшая точность анализа и классификации. В сочетании с другими метриками эта характеристика позволяет 
более точно выделять RDP-трафик среди общего потока сетевых данных.

\subsection{Признаки на основе анализа параметров TCP-соединений}

Протокол TCP предоставляет множество полей в своих заголовках, каждое из которых несет важную информацию о состоянии соединения, обмене данными и 
управлении потоком. Эти параметры играют ключевую роль в обеспечении надёжности передачи данных, но также могут служить источником ценной информации 
для анализа сетевого трафика.

В результате детального изучения особенностей сетевого взаимодействия по протоколу RDP были выявлены признаки, которые позволяют отличить данный 
тип трафика от других. Эти признаки связаны с использованием управляющих флагов TCP (PSH, ACK, SYN, FIN, RST), размером окна и другими параметрами, 
которые характеризуют поведение соединения.


\subsubsection{Вычисление частоты флагов PSH и ACK}

В контексте протокола RDP флаг PSH активно применяется для передачи пользовательских действий, таких как нажатия клавиш и движение мыши, 
а также для пересылки буферизованных изображений или звуковых данных. Наличие PSH-флагов может быть индикатором активности RDP-сессий, 
что делает их анализ полезным инструментом для идентификации такого трафика.

% Разработанная программа считает количество TCP-пакетов, у которых флаг PSH равен единице, причем она вычисляет относительно клиента и сервера.
% Другими словами, программа считает количество TCP-пакетов с установленным флагом PSH, где клиент выступает в роли получателя. Аналогично считает 
% количество TCP-пакетов с установленным флагом PSH, где сервер выступает в роли получателя. Аналогично считается количество TCP-пакетов, получаемых клиентом и сервером.

% В результате к некоторому моменту времени
% программа получает $V_{P_{src}}$ -- объем входящего трафика с установленным флагом PSH относительно клиента и $V_{P_{dest}}$ относительно сервера, а также 
% $V_{tcp}^{src}$ -- общее число входящих TCP-пакетов получаемых клиентом, $V_{tcp}^{dest}$ -- общее число входящих TCP-пакетов получаемых клиентом. 

Для анализа частоты PSH-флагов разработанная программа подсчитывает количество TCP-пакетов с установленным флагом PSH для каждой стороны соединения. 
Это позволяет выделить:

\begin{itemize}
  \item $V_{P_{init}}$: объем входящего трафика с установленным флагом PSH относительно клиента;
  \item $V_{P_{targ}}$: объем входящего трафика с установленным флагом PSH относительно сервера;
  \item $V_{tcp}^{init}$: общее количество входящих TCP-пакетов, получаемых клиентом;
  \item $V_{tcp}^{targ}$: общее количество входящих TCP-пакетов, получаемых сервером.
\end{itemize}

На основе этих данных вычисляются частоты флагов PSH для каждой стороны.

Для клиента:

\begin{equation}
  r_{psh}^{init} = \frac{V_{P_{init}}}{V_{tcp}^{init}}
\end{equation}

Для сервера:

\begin{equation}
  r_{psh}^{targ} = \frac{V_{P_{targ}}}{V_{tcp}^{targ}}
\end{equation}

Анализ частоты PSH-флагов позволяет выявить характерный для RDP трафик, где их интенсивность выше, чем у большинства других протоколов. 
Рассмотрение этих метрик как относительно клиента, так и относительно сервера важно для учёта специфики взаимодействия между сторонами: 
данные клиента часто включают команды, а данные сервера — графическую или звуковую информацию. Такой двусторонний анализ помогает точнее 
охарактеризовать тип и особенности обмена данными.

Частота флагов ACK вычисляется аналогично частоте PSH-флагов. Программа накапливает данные о числе TCP-пакетов с установленным 
флагом ACK относительно клиента ($V_{A_{init}}$) и сервера ($V_{A_{targ}}$). После этого рассчитываются частоты ACK-флагов.

Для клиента:
  \begin{equation}
    r_{ack}^{init} = \frac{V_{A_{init}}}{V_{tcp}^{init}}
  \end{equation}

Для сервера:
  \begin{equation}
    r_{ack}^{targ} = \frac{V_{A_{targ}}}{V_{tcp}^{targ}}
  \end{equation}

  Анализ частоты ACK-флагов предоставляет информацию о количестве подтверждений, отправляемых и получаемых сторонами 
  соединения. В контексте RDP трафика частота ACK может отражать характер обмена данными, где сервер часто отправляет 
  подтверждения для поступающих от клиента команд и событий.

\subsubsection{Вычисление отношения ACK/PSH}

Для более детального анализа характерных признаков RDP-сессий программа также вычисляет отношения частот ACK- и PSH-флагов 
для клиента и сервера.

Для клиента:

\begin{equation}
  r_{ack/psh}^{init} = \frac{r_{ack}^{init}}{r_{psh}^{init}}
\end{equation}

Для сервера:

\begin{equation}
  r_{ack/psh}^{init} = \frac{r_{ack}^{targ}}{r_{psh}^{targ}}
\end{equation}

Данный подход объединяет две ключевые особенности RDP-трафика: интенсивное использование PSH-флагов для передачи данных сервером и 
частое подтверждение этих данных клиентом с помощью ACK-флагов.

RDP генерирует активный двусторонний трафик, где сервер отправляет значительные объемы данных (в том числе с PSH-флагами), 
а клиент, в свою очередь, регулярно отвечает подтверждениями (ACK-флагами). Вычисление отношения ACK/PSH позволяет выявить 
баланс между передачей данных и подтверждениями, что особенно важно для идентификации интерактивных сессий.

Эта комбинация признаков позволяет более точно отличать RDP-трафик от других видов трафика, где подобное поведение не столь выражено.

\subsubsection{Вычисление разности числа исходящих и входящих ACK-флагов}

Баланс между количеством подтверждений (ACK), отправленных и полученных в рамках TCP-сессии, может отражать динамику 
взаимодействия сторон. В интерактивных протоколах, таких как RDP, где клиент активно подтверждает получение данных от 
сервера, разность числа ACK-флагов может быть индикатором асимметрии трафика. Этот показатель предоставляет дополнительную 
информацию о характере передачи данных, что полезно для анализа таких сессий.

В программе вычисляется модуль разности числа TCP-пакетов с установленным флагом ACK, причем рассматривается только относительно клиента:

\begin{equation}
  d_{ack}^{init} = | V_{A_{src}} - V_{A_{dest}} |
\end{equation}

где $V_{A_{src}}$ -- число исходящих ACK-флагов от клиента, а $V_{A_{dest}}$ -- число входящих ACK-флагов, получаемых клиентом. В данном случае необязательно
еще вычислять разность исходящих и входящих ACK-флагов относительно сервера, так как, по сути< данные величины будут одинаковы. Ведь вычисления проводятся под модулем.

Данная метрика полезна тем, что она позволяет выявить асимметрию в подтверждениях, характерную для RDP. Например, в типичной RDP-сессии 
клиент отправляет больше ACK-флагов, подтверждая получение данных от сервера. Напротив, другие протоколы могут демонстрировать иной баланс 
или меньшую интенсивность подтверждений. В сочетании с другими метриками разность числа ACK-флагов помогает уточнить модель поведения трафика 
и улучшает точность обнаружения RDP.

\subsubsection{Вычисление отношения числа флагов SYN на сумму флагов FIN + RST}


Для протоколов удалённого доступа, таких как RDP, характерно длительное и стабильное 
соединение, которое устанавливается один раз и остается активным в течение продолжительного времени. В отличие от других приложений, 
где соединения часто открываются и закрываются, RDP демонстрирует низкую частоту использования флагов FIN и RST, завершающих соединение, 
после начального установления связи с помощью SYN. Таким образом, соотношение числа флагов SYN к сумме флагов FIN и RST может быть полезным индикатором.

В программе подсчитываются TCP-пакеты с установленными флагами SYN ($V_{S_{init}}$, $V_{S_{targ}}$), FIN ($V_{F_{init}}$, $V_{F_{targ}}$) и RST 
($V_{R_{init}}$, $V_{R_{targ}}$) для клиента и сервера. После завершения временного интервала вычисляются отношения:

\begin{equation}
  \begin{aligned}
    r_{syn/fin+rst}^{init} = \frac{V_{S_{init}} + 1}{V_{F_{init}} + V_{R_{init}} + 1} \\
    r_{syn/fin+rst}^{targ} = \frac{V_{S_{init}} + 1}{V_{F_{init}} + V_{R_{init}} + 1}
  \end{aligned}
\end{equation}


Единица добавляется к числителю и знаменателю для избежания деления на ноль.

Это отношение полезно для анализа, так как в типичной RDP-сессии большая часть трафика проходит в рамках установленного соединения с 
минимальным использованием флагов FIN и RST.

\subsubsection{Вычисление среднего значения размера окна и частоты его обновления}

В ходе сеанса TCP-соединения размер окна может изменяться в зависимости от состояния сети и загруженности буфера принимающей стороны:

\begin{itemize}
  \item когда принимающая сторона запрашивает больше данных (увеличивая размер окна), это считается обновлением окна.
  \item если окно уменьшается (когда буфер заполнен), это также обновление окна.
\end{itemize}

 Программа при перехвате $n$ пакетов сохраняет значения размера окна $p_{w_1}, p_{w_2}, \dots p_{w_n}$ и вычисляет среднее значение по формуле:

 \begin{equation}
  \mu_w = \frac{1}{n} \sum_{i=1}^{n} \Delta p_{w_i}
\end{equation}

В интерактивных протоколах, таких как RDP, изменения сетевой нагрузки и взаимодействия с пользователем приводят к частым изменениям размера окна, 
поскольку требуется быстрая передача данных с минимальной задержкой. Следовательно, частота обновлений окна может помочь в определении 
протокола удаленного рабочего стола.

Данная величина будет зависеть от некоторого промежуточного интервала $\tau$ и количества изменений размера окна $k$. Таким образом частота вычисляется по формуле:

\begin{equation}
  r_w = \frac{k}{\tau}
\end{equation}

Изменения размера окна характерны для интерактивных протоколов, таких как RDP, где нагрузка на сеть и активность пользователя могут приводить к частым 
обновлениям. Высокая частота обновлений и характер изменения размера окна могут указывать на интенсивность взаимодействий, что делает эти метрики 
полезными для идентификации RDP-трафика.

\section{Модель LSTM для анализа трафика}

Для анализа сетевого трафика и выявления специфических протоколов, таких как RDP, необходимо применять подходы, которые способны обрабатывать 
последовательные данные с учетом их временной структуры. Одной из наиболее подходящих моделей для этой задачи является рекуррентная нейронная 
сеть на основе LSTM (Long Short-Term Memory).

Модель LSTM обладает уникальной способностью учитывать как кратковременные, так и долговременные зависимости в данных. Это делает ее особенно 
эффективной для анализа сетевого трафика, который состоит из последовательностей пакетов с временной корреляцией. В данном разделе описывается, 
почему именно LSTM была выбрана для задачи анализа трафика, как она реализована в программе, а также каким образом модель обучалась для выявления 
трафика RDP.




\subsection{Особенности архитектуры LSTM}
  \subsubsection{Введение в рекуррентные нейронные сети}
  Рекуррентные нейронные сети (Recurrent Neural Networks, RNN) занимают особое место в области машинного обучения 
  благодаря своей способности обрабатывать последовательные данные. В отличие от традиционных нейронных сетей, которые 
  рассматривают входные данные как независимые и статичные, RNN способны учитывать контекст предыдущих шагов, что делает их незаменимыми 
  для анализа временных рядов, текстов, сигналов и других данных с временной структурой.  

  Ключевая особенность RNN заключается в наличии петли обратной связи, которая позволяет передавать информацию о предыдущих состояниях на последующие 
  шаги. Это позволяет сети обучаться выявлять зависимости во времени, что особенно важно для анализа сетевого трафика, где порядок и временные связи 
  между событиями играют критическую роль.  

  Однако классические RNN сталкиваются с проблемой обучения на длинных последовательностях из-за эффекта <<затухания градиентов>>. В свою очередь, градиентом 
  называется производная функции ошибки по параметрам модели. В процессе обратного распространения ошибки (backpropagation) градиенты используются 
  для корректировки весов нейронов таким образом, чтобы минимизировать ошибку предсказаний модели.
  Когда ошибка распространяется обратно через множество слоев или временных шагов, её величина может значительно уменьшаться. Это приводит к тому, что 
  веса при обновлении изменяются на слишком малые значения, и обучение проходит неэффективно или останавливается, то есть алгоритм обучения не сходится \cite{grad}. 
  Эта проблема затрудняет захват долгосрочных зависимостей, что ограничивает их применение в сложных задачах. 
  
  Для решения этой проблемы была предложена 
  архитектура Long Short-Term Memory (LSTM) немецкими исследователями Зеппом Хохрайтером (Sepp Hochreiter) и Юргеном Шмидхубером (Jürgen Schmidhuber) в 1997 году.
  Их работа, опубликованная в статье <<Long Short-Term Memory>>, стала основой для дальнейшего развития рекуррентных нейронных сетей и нашла широкое 
  применение в самых разнообразных областях, включая обработку естественного языка, распознавание речи и анализ временных рядов.

  Благодаря введению специальных механизмов управления памятью LSTM способна сохранять информацию на протяжении долгих временных интервалов, что сделало эту 
  модель чрезвычайно эффективной для работы с последовательными данными.

  В данной работе основное внимание будет уделено архитектуре LSTM, так как она является наиболее подходящей для анализа сетевого трафика, где важно учитывать как 
  краткосрочные, так и долгосрочные зависимости между событиями. LSTM позволяет извлечь максимальную информацию из временных метрик, обеспечивая высокую 
  точность в задачах идентификации протоколов, таких как RDP.

  \subsubsection{Структура ячейки LSTM}

  Основной элемент модели -- ячейка LSTM, которая состоит из нескольких ключевых компонентов, взаимодействующих через специальные механизмы управления 
  потоком информации.

  Каждая ячейка LSTM содержит следующие основные элементы:  

  \begin{enumerate}
    
    \item Забывающий вентиль (Forget Gate).
    
    Этот вентиль определяет, какая информация из состояния памяти на предыдущем шаге ($C_{t-1}$) должна быть забыта.  
    Это важно для предотвращения <<загрязнения>> памяти нерелевантной информацией.  
    
    Формула забывающего вентиля:  
    \begin{equation}      
        f_t = \sigma(W_{xf}x_t + W_{hf}h_{t-1} + b_f),
    \end{equation}
     где $x_t$ -- входной вектор (входные данные на текущем шаге), $h_{t-1}$ -- скрытое состояние с предыдущего шага, $W_{xf}$ и $W_{hf}$ -- матрицы весов,
     $b_i$ -- смещение, $\sigma$ -- сигмоида, которая сжимает значения в диапазон $[0, 1]$, задавая <<вес>> забывания.

    \item Входной вентиль (Input Gate).
    
    Данная компонента управляет количеством новой информации, поступающей в ячейку, а также решает, 
    какую часть новой информации из входных данных \(x_t\) и скрытого состояния \(h_{t-1}\) 
    следует записать в память.

    Входной вентиль определяется следующей формулой:  

    \begin{equation}
        i_t = \sigma(W_{xi}x_t + W_{hi}h_{t-1} + b_i).
    \end{equation}
    
    Стоит отметить, что параллельно создаётся <<кандидат>> (Candidate state) новой информации (\(\tilde{C}_t\)):  
    \begin{equation}
      \tilde{C}_t = \tanh(W_{xC}x_t + W_{hC}h_{t-1} + b_C).      
    \end{equation}

    \item Обновление состояния ячейки (Cell state). 
    После работы забывающего и входного вентилей состояние памяти (ячейки) обновляется:  
    \begin{equation}
        C_t = f_t \odot C_{t-1} + i_t \odot \tilde{C}_t,
    \end{equation}
    где \(\odot\) обозначает поэлементное произведение.  
    
    Таким образом, модель сохраняет важную информацию из прошлого (\(f_t \odot C_{t-1}\)) и добавляет новую информацию (\(i_t \odot \tilde{C}_t\)).
 
    \item Выходной вентиль (Output Gate). 
    
    Этот вентиль управляет тем, какая информация из текущего состояния ячейки должна быть передана в следующее 
    состояние или использована для предсказания.  
    
    Формула выходного вентиля:  
    \begin{equation}
        o_t = \sigma(W_{xo}x_t + W_{ho}h_{t-1} + b_o),
    \end{equation}
     где $o_t$ -- выход выходного вентиля.  
    
    \item Скрытое состояние (Hidden State).  
   
    Состояние $h_t$ Представляет информацию, используемую для выходного значения ячейки. Оно вычисляется как:  
    \begin{equation}
        h_t = o_t \odot \tanh(C_t)
    \end{equation}
  
  \end{enumerate}


  Каждый шаг работы LSTM можно описать следующим образом:  

  \begin{enumerate}
    \item Сначала забывающий вентиль (\(f_t\)) удаляет ненужную информацию из состояния памяти.  
    \item Затем вентиль записи (\(i_t\)) решает, какую новую информацию добавить, и создаёт кандидата новой информации (\(\tilde{C}_t\)).  
    \item Обновляется состояние памяти (\(C_t\)) с учётом старой и новой информации.  
    \item Выходной вентиль (\(o_t\)) управляет тем, какая часть информации из памяти используется для обновления скрытого состояния (\(h_t\)).  
    \item Скрытое состояние (\(h_t\)) передаётся дальше в цепочке, одновременно выступая как выход текущей ячейки.  
  \end{enumerate}

  Таким образом, LSTM эффективно решает проблему долговременной зависимости, обеспечивая контроль над сохранением, забыванием и использованием информации.

\subsubsection{Преимущества LSTM перед другими моделями}

  LSTM (Long Short-Term Memory) -- одна из наиболее подходящих моделей для анализа последовательных данных, таких как 
  сетевой трафик, благодаря ряду её ключевых преимуществ: 

  \begin{enumerate}
    \item Учёт временных зависимостей. LSTM обладает механизмом памяти, позволяющим учитывать как краткосрочные, так и 
    долгосрочные зависимости. В анализе сетевого трафика это крайне важно, так как порядок и время между событиями могут 
    свидетельствовать о типе используемого протокола. Например, RDP имеет специфические временные закономерности и последовательности пакетов, 
    которые LSTM может захватывать.  

    \item Устойчивость к затуханию градиентов. Благодаря использованию специальных элементов управления (входной, выходной и забывающий вентили), 
    LSTM справляется с проблемой затухания градиентов, характерной для классических RNN. Это делает модель устойчивой при работе с длинными 
    временными последовательностями, что критично при анализе протяжённых сессий.  

    \item Способность обрабатывать данные с разной длиной последовательностей. В сетевом трафике разные сессии могут иметь разное количество
    пакетов и временных интервалов. LSTM может обрабатывать такие данные, не требуя строгой унификации длины, что делает её гибкой и адаптивной.  

    \item Эффективная работа с метриками. Модель способна извлекать скрытые паттерны из временных метрик, таких как частота флагов, 
    размер окна, джиттер и другие. Эти метрики становятся входными признаками, которые LSTM анализирует, чтобы классифицировать протоколы.  

  \end{enumerate}

  Хотя LSTM является одним из лучших вариантов для анализа сетевого трафика, существуют и другие модели, которые тоже могут быть 
  полезны в зависимости от конкретной задачи:

  \begin{enumerate}
    \item GRU (Gated Recurrent Unit). GRU -- это упрощённая версия LSTM с меньшим количеством параметров. Она тоже эффективно справляется 
    с захватом временных зависимостей, но менее выразительна, что может быть недостатком в сложных задачах. GRU потребляет меньше ресурсов, 
    поэтому может быть полезна для более простых задач анализа сетевого трафика.  

    \item 1D-CNN (Одномерные свёрточные нейронные сети). Свёрточные нейронные сети могут использоваться для анализа временных рядов, 
    извлекая пространственно-временные паттерны. Они хорошо справляются с локальными зависимостями и могут быть быстрее в обучении, 
    но не учитывают долгосрочные зависимости так эффективно, как LSTM.  

    \item Transformer-based модели. Современные трансформеры, такие как BERT или их модификации для временных данных, превосходят RNN 
    в захвате долгосрочных зависимостей. Однако они требуют больше вычислительных ресурсов и больших объёмов данных для обучения. Это 
    может быть излишним для задач анализа трафика.  

    \item Random Forest и Gradient Boosting. Если задать задачу не как анализ временных рядов, а как классификацию на основе статистик,
    эти алгоритмы машинного обучения тоже могут быть применимы. Однако они не учитывают порядок и временные зависимости, что снижает их
    эффективность в задачах анализа последовательных данных.  
  \end{enumerate}


  Среди перечисленных моделей LSTM оказалась оптимальным выбором для задачи анализа сетевого трафика, так как она сочетает:
  \begin{itemize}
    \item способность работать с временными и последовательными данными,  
    \item устойчивость к шуму, характерному для сетевых данных,  
    \item точность при работе с большим числом временных метрик.  
  \end{itemize}

  Стоит отметить, что если задача идентификации RDP будет расширяться, и потребуется больше вычислительной мощности или улучшенная работа с контекстом, 
  то можно будет рассмотреть применение трансформеров или их комбинации с LSTM.

\subsection{Применение LSTM в задаче анализа трафика}

В данном подразделе рассматривается реализация модели LSTM для задачи анализа сетевого трафика, а именно выявления протокола RDP среди общего объёма 
данных. Модель была реализована с использованием языка программирования Python и библиотек Keras. Ниже представлено описание модели и её структуры.

Модель LSTM была определена следующим образом:

\begin{lstlisting}[language=Python, caption=Определение модели LSTM]
  def define_model(self):
    self.model = Sequential()
    self.model.add(LSTM(units=64, return_sequences=True))
    self.model.add(Dense(units=32, activation='relu'))
    self.model.add(Dense(units=2, activation='softmax'))
    self.model.compile( optimizer='adam'
                      , loss='categorical_crossentropy'
                      , metrics=['accuracy'] )
\end{lstlisting}

Модель состоит из трёх основных слоёв:
\begin{itemize}
    \item Входной слой LSTM. Слой определён с помощью функции \textit{LSTM} из библиотеки Keras, в которой количество нейронов 
    установлено равным 64. Выбор такого количества обусловлен необходимостью балансировать между вычислительными затратами и способностью 
    модели захватывать временные зависимости в сетевом трафике. Параметр \textit{return\_sequences=} \textit{True} позволяет передавать последовательности 
    данных следующему слою, что важно для анализа временных зависимостей.
    
    \item Полносвязный слой для классификации. В данном слое используются 32 нейрона с функцией активации ReLU (\textit{Rectified Linear Unit}). 
    Эта функция активации обеспечивает линейность при положительных значениях входного сигнала, что позволяет нейронам эффективно обрабатывать данные и 
    минимизировать проблему исчезающего градиента. Полносвязный слой преобразует выходы LSTM в более компактное представление, что позволяет выделить 
    наиболее значимые признаки и передать их в следующий слой. Это упрощает задачу классификации, снижая размерность данных.
    
    \item Выходной слой. Содержит 2 нейрона, что соответствует двум классам: \([1, 0]\) для протокола RDP и \([0, 1]\) для всего остального 
    трафика. В качестве функции активации используется \textit{softmax}, которая интерпретирует выходные значения как вероятности принадлежности входных 
    данных к каждому из классов.
\end{itemize}

Модель была оптимизирована с помощью алгоритма \textit{adam}, который является эффективным методом стохастической оптимизации, адаптирующим скорость 
обучения для каждого параметра. В качестве функции потерь используется \textit{categorical crossentropy}, используется для многоклассовой классификации, 
поскольку она сравнивает распределение вероятностей, возвращаемое моделью, с истинными метками.

Стоит отметить, что все слои и параметры были выбраны с учётом специфики задачи анализа трафика.

В LSTM-слое выбрано 64 нейрона, так как этого оказалось достаточно для моделирования сложных временных зависимостей в данных сетевого трафика. Полносвязный 
слой имеет 32 нейрона для снижения размерности и увеличения способности модели выделять важные признаки. Совокупность использования слоя LSTM и Dense 
позволяет модели выделять наиболее важные признаки, характерные для классов данных.

Таким образом, представленная модель LSTM образует гибкую и мощную архитектуру для анализа сетевого трафика и выявления протокола RDP. В следующем 
разделе будет детально описан функционал программы, эксплуатирующей модель нейронной сети LSTM. 
  

\section{Программная реализация}

В рамках последней курсовой работы, посвящённой теме <<Статистический анализ сетевого трафика для обнаружения активной RDP-сессии>>, была разработана 
программа, предоставляющая следующие функциональные возможности:

\begin{enumerate}
  \item Перехват трафика. Программа позволяла пользователю выполнять перехват сетевого трафика в одном из двух режимов:
    \begin{itemize}
      \item В первом режиме осуществлялся перехват всего сетевого трафика с выводом информации о каждом перехваченном пакете в консоль.
      \item Во втором режиме вывод ограничивался только пакетами, содержащими признаки активной RDP-сессии. Определение таких признаков осуществлялось на 
      основе статистических методов анализа.
    \end{itemize}
  
  \item Запись данных в файл. После завершения сбора трафика предоставлялась возможность сохранить данные в файл. В файле хранилась информация о 
  каждом пакете, включая все ключевые поля, необходимые для дальнейшего анализа.

  \item Считывание данных из файла. Программа позволяла анализировать ранее собранный сетевой трафик без необходимости повторного перехвата. Однако 
  корректное считывание данных обеспечивалось только для файлов, созданных с помощью этой же программы.

  \item Анализ сетевого трафика. Программа реализовывала механизм просмотра активных сессий, где сессия считалась активной, если обмен пакетами между 
  двумя компьютерами продолжался более 10 секунд. Это решение исключало из анализа кратковременные сессии, в которых происходил обмен лишь несколькими пакетами. 
  Сессия определялась как пара IP-адресов и общий порт, по которому осуществлялся обмен трафиком.  
  Пользователь мог получить подробную информацию о сетевом трафике, связанном с конкретным IP-адресом, а также построить графики, основанные на различных 
  методах анализа.
\end{enumerate}

На основе вышеописанной программы <<traffic-detection.py>>, использовавшейся для анализа сетевого трафика и выявления признаков RDP, была разработана новая 
версия программы. Эта версия существенно расширяет функционал и обеспечивает интеграцию с моделью LSTM для автоматического обнаружения активных RDP-сессий. 
Структура новой программы будет подробно рассмотрена в следующих разделах.

\subsection{Общая структура программы}

Для интеграции модели нейронной сети в программу <<traffic-detection.py>> было принято решение провести полный рефакторинг исходной логики программы. 
Исходно вся функциональность находилась в одном файле <<traffic-detection.py>>, что ограничивало гибкость и масштабируемость. После тщательного 
пересмотра и рефакторинга, была добавлена логика для работы с нейронными сетями, что позволило значительно улучшить эффективность программы.

Помимо пересмотра логики, в новой версии программы были внесены следующие изменения:

\begin{itemize}
  \item Изменение алгоритмов классификации пакетов. Одной из ключевых задач при анализе сетевого трафика является правильная 
  классификация пакетов. В старой версии программы для классификации пакетов использовался алгоритм, который создавал структуры, называемые <<сессиями>>. 
  Эти сессии связывались с конкретными пакетами, основанными на информации о взаимодействии между двумя устройствами через TCP-протокол. Процесс был 
  основан на принципе <<трехстороннего рукопожатия>> (Three-way Handshake), который включает следующие этапы:
  \begin{enumerate}
    \item Инициация соединения: клиент отправляет серверу пакет с флагом SYN, предлагая начать соединение.
    \item Ответ сервера: сервер подтверждает получение запроса, отправляя клиенту пакет с флагами SYN и ACK.
    \item Завершение соединения: клиент подтверждает установление связи, отправив пакет с флагом ACK.
  \end{enumerate}

  Программа отслеживала эти этапы и фиксировала активные сессии, создавая соответствующие структуры. В новой версии программы логика изменилась. 
  Вместо создания сессий на основе трехстороннего рукопожатия программа сохраняет в структуру пару IP-адресов (клиента и сервера), пару портов и время 
  перехвата. При перехвате новых пакетов программа проверяет IP-адреса и порты, сопоставляя их с уже существующими структурами или создавая новые. Это 
  позволило значительно упростить алгоритм и повысить его эффективность, что было также достигнуто благодаря добавлению многопоточности для параллельной 
  обработки пакетов.

  \item Добавление динамического определения общего порта. Общий порт -- это порт получателя, указанный в первом пакете при установке соединения. 
  В ходе тестирования программы было выявлено, что в процессе активной сессии клиент или сервер могут изменять свои порты передачи данных, в то время как 
  порт получателя остаётся неизменным. Это явление наблюдается в протоколах RDP и HTTP/HTTPS, где данные могут передаваться через несколько портов 
  одновременно. Для эффективного отслеживания таких случаев был разработан алгоритм динамического определения общего порта. Это решение позволяет 
  программе правильно отслеживать пакеты, даже если клиент или сервер меняют порты, но сохраняют общий порт получателя.

  \item Изменение логики обнаружения признаков RDP-сессии. Ранее программа для выявления признаков RDP-сессий использовала статистический анализ 
  на основе расчета различных параметров сетевого трафика, таких как время между пакетами, размер пакетов и другие метрики. Каждые 5 секунд программа 
  вычисляла параметры для каждой активной сессии и на основе пороговых значений принимала решение о наличии признаков RDP. В новой версии программы 
  функцию выявления RDP-сессий теперь выполняет нейронная сеть, обученная на метриках, основанных на статистических признаках трафика. Как именно 
  нейронная сеть осуществляет классификацию и как она была интегрирована в программу, будет подробно рассмотрено далее.

  \item Изменение построения графиков и добавление новых. В старой версии программы анализ сетевого трафика осуществлялся с помощью построения 
  графиков, которые визуализировали различные параметры, такие как задержка, частота флагов и другие метрики. На основе этих графиков выявлялись потенциальные 
  признаки RDP-сессий. В новой версии программы были добавлены дополнительные графики для улучшения анализа, а также для более детального отслеживания динамики 
  сетевого трафика и его связи с активностью RDP.
\end{itemize}

Эти изменения обеспечили более эффективную работу программы и улучшили её способность к анализу сетевого трафика, позволяя выявлять признаки 
активных RDP-сессий с высокой точностью. Программа теперь использует современные методы обработки данных, включая нейронные сети, что делает 
её более гибкой и мощной для решения поставленной задачи.

\subsection{Процесс обучения модели LSTM и ее использование в программе}

Основная идея использования нейронной сети заключается в том, чтобы с фиксированным временным интервалом $\tau$ определять, 
является ли каждая активная сессия RDP. Для этого было выбрано значение $\tau = 15$ секунд. Таким образом, каждые 15 секунд 
программа анализирует накопленные данные, вычисляет метрики и выполняет предсказание для всех активных сессий. Это позволяет 
своевременно идентифицировать RDP-сессии в потоке сетевого трафика.  

Однако для того чтобы использовать модель нейронной сети в выявлении RDP-трафика, первым этапом необходимо было её обучить. Для этого была проведена 
серия экспериментов по сбору сетевого трафика с различных устройств и протоколов.  

Собранные данные сохранялись в файлы с помощью переработанной программы. Самой сложной частью оказался процесс разметки данных. Благодаря сохранённой 
информации о временных метках каждого перехваченного пакета удалось восстановить последовательность событий в сетевом трафике. Данные были разбиты на 
пятнадцатисекундные интервалы, в рамках которых рассчитывались ключевые метрики, необходимые для обучения модели.  

В результате анализа сетевого трафика было разработано 21 метрика, наиболее значимая для выявления RDP-сессий. Каждые 15 секунд для каждой активной сессии 
происходит вычисление:  

\begin{enumerate}
  \item средней задержки;
  \item стандартного отклонения временных интервалов между пакетами;
  \item среднего джиттера;
  \item медианы временных интервалов;
  \item отношения объема входящего на исходящий трафик относительно клиента;
  \item отношения объема входящего на исходящий трафик относительно сервера;
  \item отношения объема UDP-трафика и TCP-трафика;
  \item среднего значения объема пакетов получаемого относительно клиента;
  \item среднего значения объема пакетов получаемого относительно сервера;
  \item частоты флагов PSH относительно клиента;
  \item частоты флагов PSH относительно сервера;
  \item частоты флагов ACK относительно клиента;
  \item частоты флагов ACK относительно сервера;
  \item отношения ACK/PSH относительно клиента;
  \item отношения ACK/PSH относительно сервера;
  \item разности числа исходящих и входящих ACK-флагов относительно клиента;
  \item отношения количество флагов SYN на сумму флагов FIN + RST относительно клиента;
  \item отношения количество флагов SYN на сумму флагов FIN + RST относительно сервера;
  \item среднего значения размера окна;
  \item частоты обновления размера окна;
  \item количества пакетов.
\end{enumerate}

После вычисления этих метрик для каждой активной сессии формируется вектор $x_t$, который подаётся на вход нейронной сети. 
Эти векторы сохраняются в файл, причём каждому из них присваивается метка <<RDP>> или <<не RDP>>. Разметка данных позволила 
подготовить качественный набор данных для последующего обучения.  

Обучение модели проводится на размеченных данных, после чего обученная модель сохраняется в файл формата \textit{.keras}. Основная программа 
загружает эту модель с помощью метода \textit{load_model} из библиотеки \textit{keras.models}. После загрузки модель переключается в 
режим предсказания, что позволяет ей обрабатывать данные в реальном времени. Используя метод \textit{predict()}, каждые 15 секунд модель 
анализирует метрики активных сессий и делает предсказания об их принадлежности к RDP.  

\subsection{Демонстрация работы программы}

Тестирование программы проводилось на виртуальных машинах (ВМ), работающих с различными операционными системами и подключённых к реальной локальной сети. 
Для эксперимента использовались следующие системы: Windows 10 Professional версии 22H2 (обозначим как Win), Kali Linux 2024.3 Release (Kali) и 
Ubuntu 22.04.5 LTS (Ubuntu). Эксперимент заключался в установке соединений между двумя ВМ, использующими различные протоколы, и запуске программы 
с обученной моделью нейронной сети на третьей ВМ для анализа сетевого трафика.

Особенностью эксперимента стало то, что поведение протокола RDP при соединении между ВМ с разными операционными системами оказалось различным. Это 
повлияло на показатели признаков и метрик, используемых моделью для классификации трафика. Поэтому для повышения точности обучения и предсказаний в 
модели учитывались особенности взаимодействия различных операционных систем.

В этом разделе будет продемонстрирована работа программы с разными видами прикладных протоколов и проведен сравнительный анализ поведения RDP-сессий 
при взаимодействии разных ОС и альтернативных программ удаленного доступа.

\subsubsection{Сравнение поведения прикладных протоколов}

Для начала был запущен перехват трафика с фильтром RDP, как показано на рисунке \ref{main1}. То есть сейчас включен режим, когда в консоль 
будет выводиться активные RDP-сессии. Стоит отметить, что при перехвате теперь можно заметить строки прогресса. Они символизируют о том, что
уже прошло 7 раз было совершено предсказание активных сессий. То есть с момента запуска программы прошло уже 105 секунд.

\begin{figure}[H]
  \centering
  \includegraphics[width=0.9\textwidth]{pics/main-view.png}
  \caption{Запуск перехвата трафика с фильтром RDP}
  \label{main1}
\end{figure}

На ВМ Kali был предварительно запущен HTTP-сервер, на котором хранилось несколько файлов разного размера, а также видео которое также можно посмотреть.
Просмотр страницы HTTP-сервера совершался с другой ВМ типа Win. На Win совершались активные попытки скачивания большого файла с сервера. Из рисунка \ref{http1}
видно, что ложных срабатываний за весь период активного взаимодействия с HTTP сервером не наблюдалось.

\begin{figure}[H]
  \centering
  \includegraphics[width=0.9\textwidth]{pics/2http.png}
  \caption{Состояние программы при HTTP-трафике}
  \label{http1}
\end{figure}


Далее был рассмотрен еще один протокол прикладного уровня, а именно протокол FTP. Этот протокол полезен для загрузки и выгрузки файлов. Подключение осуществлялось 
по протоколу FTP между Win и Kali. После успешной установки соединения пользователм были сделаны различные команды, после того как пользователь нашел большой файл, то
он начал его скачивать, как показано на рисунке \ref{ftp1}.

\begin{figure}[H]
  \centering
  \includegraphics[width=0.9\textwidth]{pics/3ftp.png}
  \caption{Соединение по протоколу FTP}
  \label{ftp1}
\end{figure}

После нескольких успешных попыток скачивания большого файла ложных срабатываний также не наблюдалось.


\begin{figure}[H]
  \centering
  \includegraphics[width=0.9\textwidth]{pics/4ftp.png}
  \caption{Состояние программы при FTP-трафике}
  \label{ftp2}
\end{figure}

На следующих двух рисунках показаны графики одной из метрик. 



\begin{figure}[H]
  \centering
  \includegraphics[width=0.9\textwidth]{pics/http-in-out.png}
  \caption{График отношения объема входящего и исходящего трафиков (HTTP)}
  \label{httpg1}
\end{figure}

Из второго графика следует, что при FTP-сессии клиент и сервер при отправке и при получении получают почти одинаковое 
количество пакетов.


\begin{figure}[H]
  \centering
  \includegraphics[width=0.9\textwidth]{pics/ftp-in-out.png}
  \caption{График отношения объема входящего и исходящего трафиков (FTP)}
  \label{ftpg1}
\end{figure}


На следующем рисунке показано уже подключение по RDP. Соединение устанавливается между Win (клиент) и Kali (сервер). 
Стоит отметить, что для осуществления такого подключения на Kali запускался сервис XRDP, бесплатный протокол 
удаленного доступа, основанный на протоколе RDP (Microsoft Remote Desktop). Из рисунка \ref{rdp1} видно, что программа
нашла активную RDP-сессию. При подключении по RDP совершались движения мышкой и нажатия клавиш клавиатуры. Буквально после 30 секунд
модель смогла определить RDP-трафик.

\begin{figure}[H]
  \centering
  \includegraphics[width=0.9\textwidth]{pics/7rdp.png}
  \caption{Состояние программы при RDP-трафике}
  \label{rdp1}
\end{figure}

Также стоит рассмотреть следующий график, показанный на рисунке \ref{rdpg1}. Если сравнить RDP с протоколами HTTP и FTP по данной 
метрике, то видно насколько их поведение различается.

\begin{figure}[H]
  \centering
  \includegraphics[width=0.9\textwidth]{pics/rdp-in-out.png}
  \caption{График отношения объема входящего и исходящего трафиков (RDP)}
  \label{rdpg1}
\end{figure}

Далее был рассмотрен еще один протокол прикладного уровня -- SSH. На следующем рисунке показано состояние программы при активной работе
с протоколом SSH.

\begin{figure}[H]
  \centering
  \includegraphics[width=0.9\textwidth]{pics/5ssh.png}
  \caption{Состояние программы при SSH-трафике}
  \label{ssh1}
\end{figure}


Стоит отметить, что если исследовать графики, то поведение SSH и RDP достаточно похоже. Особенно если при подключении по SSH
выполнять какие-либо непрерывные действия, например открыть большой файл и листать его строки. Если сравнить графики SSH и RDP по метрике
отношения объема входящего и исходящего трафиков, то можно заметить некоторую схожесть

\begin{figure}[H]
  \centering
  \includegraphics[width=0.9\textwidth]{pics/ssh-in-out.png}
  \caption{График отношения объема входящего и исходящего трафиков (SSH)}
  \label{sshg1}
\end{figure}

Однако, если смотреть по остальным признакам, как показано на следующих двух рисунках, то можно заметить небольшое различие в этих двух видах протоколов.

\begin{figure}[H]
  \centering
  \includegraphics[width=0.9\textwidth]{pics/ssh-cnt-pack.png}
  \caption{График отображения количества входящих пакетов (SSH)}
  \label{sshg2}
\end{figure}


\begin{figure}[H]
  \centering
  \includegraphics[width=0.9\textwidth]{pics/rdp-cnt-pack.png}
  \caption{График отображения количества входящих пакетов (RDP)}
  \label{rdpg2}
\end{figure}


\subsubsection{Сравнительный анализ альтернативных программ удаленного доступа}

В этом разделе проводится сравнительный анализ альтернативных программ удаленного 
доступа, таких как Ammyy Admin, Remmina и RAdmin. Целью данного раздела является демонстрация работы программы 
при использовании этих приложений.


Для проведения эксперимента на виртуальной машине с операционной системой Windows был изменён порт по умолчанию для 
подключения через RDP. Вместо стандартного порта 3389 использовался порт 53389. На рисунке \ref{rdp2} отображено создание 
подключения по RDP и вывод программы в консоли информации о пакетах данной сессии. Программа успешно распозналa RDP-сессию, 
поскольку обученная модель не проводит анализ на основе портов. Как упоминалось ранее, такой метод анализа считается ненадёжным.

\begin{figure}[H]
  \centering
  \includegraphics[width=0.9\textwidth]{pics/8rdp.png}
  \caption{Состояние программы при RDP-трафике (смена порта)}
  \label{rdp2}
\end{figure}

Ammyy Admin -- это программа удаленного доступа и администрирования, разработанная компанией Ammyy Group. Она 
позволяет пользователям получать удаленный доступ к компьютерам и серверам через интернет и управлять ими 
в реальном времени. На следующем рисунке показано подключение с помощью Ammyy Admin между двумя ВМ операционной системы Windows. 
Так как данное приложение поддерживает протокол RDP, то программа смогла распознать признаки RDP-сессии.


\begin{figure}[H]
  \centering
  \includegraphics[width=0.9\textwidth]{pics/9ammyy.png}
  \caption{Состояние программы при RDP-трафике (использование Ammyy Admin)}
  \label{ammyy1}
\end{figure}

Аналогичное поведение наблюдается при использовании Remmina, бесплатной программы для удаленного доступа, ориентированной 
на пользователей операционной системы Linux. На рисунке \ref{remmina1} видно, как было установлено подключение между Ubuntu 
(клиента) и Win (сервера).

\begin{figure}[H]
  \centering
  \includegraphics[width=0.9\textwidth]{pics/10remmina.png}
  \caption{Состояние программы при RDP-трафике (использование Remmina)}
  \label{remmina1}
\end{figure}


RAdmin -- это программа удалённого администрирования для платформ Windows, разработанная компанией Famatech. Она 
является одной из альтернатив подключения к удаленному рабочему столу. Исходя из следующего рисунка, подключение с помощью RAdmin
также распознается программой.


\begin{figure}[H]
  \centering
  \includegraphics[width=0.9\textwidth]{pics/11radmin.png}
  \caption{Состояние программы использовании RAdmin}
  \label{radmin1}
\end{figure}

Модель LSTM продемонстрирована как эффективный инструмент для выявления протокола RDP вне зависимости от конфигурации 
сетевых соединений. Данный эксперимент подчеркивает важность использования обученных моделей для точного анализа сетевого трафика.

% \subsection{Точность и производительность модели}



\conclusion
  
В ходе выполнения данной работы была разработана программа для автоматического выявления RDP-сессий в сетевом трафике с 
использованием нейронных сетей, в частности модели LSTM. Данная работа позволила углубленно изучить особенности протокола 
RDP и выявить его характерные признаки в сетевом трафике. На основе этих знаний были разработаны метрики для анализа и 
классификации RDP-трафика, что позволило эффективно использовать возможности модели LSTM для решения задачи классификации.

Тестирование показало, что разработанная система способна точно определять наличие RDP-соединений среди общего потока сетевых 
данных, что делает её полезным инструментом для обеспечения информационной безопасности.

Использование нейронных сетей для анализа сетевого трафика открывает новые перспективы в области защиты информации и борьбы 
с кибератаками, особенно в контексте современных цифровых технологий.


\begin{thebibliography}{15}
    \bibitem{lib1}
    Фаткиева, Р. Р Разработка метрик для обнаружения атак на основе анализа сетевого трафика / Р. Р. Фаткиева // Вестник Бурятского государственного университета, выпуск 9, 2013. С. 81-86.
    \bibitem{lib2}
    Карачанская, Е.В Метод выявления аномалий сетевого трафика, основанный на его самоподобной структуре / Е.В. Карачанская, Н.И. Соседова // Безопасность информационных технологий, том 26, №1, 2019. С. 98-110.
    \bibitem{2}
    Alexey Удалённый рабочий стол RDP: как включить и как подключиться по RDP [Электронный ресурс] : настройка удаленного рабочего стола RDP. -- URL: https://hackware.ru/?p=11835\#11 (дата обращения: 31.03.2023). -- Загл. с экрана. -- Яз. рус.
    \bibitem{rdp2}
    https://learn.microsoft.com/en-us/troubleshoot/windows-server/remote/understanding-remote-desktop-protocol
    \bibitem{stack}
    https://cyberleninka.ru/article/n/semiurovnevaya-model-osi-iso-i-stek-protokolov-tcp-ip-issledovanie-vzaimootnosheniya-i-interpretatsii/viewer
    % \bibitem{rdp2}
    % Remote Utilities RDP [Электронный ресурс] : подключение к удаленному рабочему столу. --  Remote Utilities Pty (Cy) Ltd 2010 -- 2023. -- URL:  https://www.remoteutilities.com/support/docs/rdp/ (дата обращения: 31.03.2023). -- Загл. с экрана. -- Яз. англ.
    \bibitem{osi-model}
    Maintenance script Модель OSI [Электронный ресурс] : описание модели OSI. -- Университет ИТМО. -- URL: http://neerc.ifmo.ru/wiki/index.php?title=OSI_Model (дата обращения: 31.03.2023). -- Загл. с экрана. --  Яз. рус.
    \bibitem{winsize}
    https://learn.microsoft.com/ru-ru/troubleshoot/windows-server/networking/description-tcp-features
    \bibitem{tcpflags}
    KeyCDN TCP flags [Электронный ресурс] : описание TCP-флагов. -- proinity LLC 2023. -- URL: https://www.keycdn.com/support/tcp-flags\#:$\sim$:text=ACK (дата обращения: 31.03.2023). -- Загл. с экрана. --  Яз. англ.
    \bibitem{udpseg}
    https://learn.microsoft.com/en-us/openspecs/windows_protocols/ms-rdpeudp/2744a3ee-04fb-407b-a9e3-b3b2ded422b1
    \bibitem{grad}
    https://neerc.ifmo.ru/wiki/index.php?title=Проблемы_нейронных_сетей
    \bibitem{rnn}
    Осовский С. Нейронные сети для обработки информации. -- Финансы и статистика, 2004.
    \bibitem{keras}
    https://keras.io/api/layers/recurrent_layers/lstm/
    \bibitem{lstm1}
    Hochreiter S. Long Short-term Memory //Neural Computation MIT-Press. -- 1997. https://blog.xpgreat.com/file/lstm.pdf
    \bibitem{lstm2}
    https://colah.github.io/posts/2015-08-Understanding-LSTMs/
    % \bibitem{socket1}
    % The Python Standard Library Socket - Low-level networking interface [Электронный ресурс] : описание методов библиотеки socket. -- Python Software Foundation 2001 -- 2023. -- URL: https://docs.python.org/3/library/socket.html (дата обращения: 31.03.2023). -- Загл. с экрана. -- Яз. англ.
    % \bibitem{session}
    % How Terminal Services Works [Электронный ресурс] : описание службы терминалов удаленного рабочего стола. -- Microsoft 2023. -- URL:  https://learn.microsoft.com/en-us/previous-versions/windows/it-pro/windows-server-2003/cc755399(v=ws.10)?redirectedfrom=MSDN (дата обращения: 14.04.2023). -- Загл. с экрана. -- Яз. англ.
    % \bibitem{rdp1}
    % Antenore Gatta The fast, stable, and always free Linux RDP client [Электронный ресурс] : подключение по RDP с помощью приложения Remmina. -- URL: https://www.remmina.org/remmina-rdp/ (дата обращения: 15.04.2023). -- Загл. с экрана. -- Яз.  англ.
    % \bibitem{rdpport}
    % Jenks, A. Change the listening port for Remote Desktop on your computer / A. Jenks, S. Manheim, H. Lohr // [Электронный ресурс] : изменение порта прослушивания для RDP. -- Microsoft, 2023. -- URL: https://learn.microsoft.com/ru-ru/windows-server/remote/remote-desktop-services/clients/change-listening-port (дата обращения: 28.04.2023). -- Загл. с экрана. -- Яз. англ.
    % \bibitem{dev0}
    % Киреева, Н. В Исследование трафика локальной сети посредством сетевого анализатора «Wireshark» [Электронный ресурс] : методическая разработка к лабораторной работе / Н. В. Киреева, В.П. Зайкин, Н. Н. Васин // Поволжский государственный университет телекоммуникаций и информатики, Самара, 2010. -- URL: http://ib.psuti.ru/content/metod/методическиеWireshark.pdf (дата обращения 04.05.2023). -- Загл. с экрана. -- Яз. рус.
    % \bibitem{dev1}
    % Standard deviation [Электронный ресурс] : описание нахождения стандартного отклонения. -- Википедия. -- URL: https://en.wikipedia.org/wiki/Standard_deviation (дата обращения 04.05.2023). -- Загл. с экрана. -- Яз. англ.
  \end{thebibliography}

  \appendix

    \section{Код main.py}
    % \inputminted{py}{src/code/main.py}


\end{document}